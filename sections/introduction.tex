%! TeX root: ../main.tex
\section{Introduction}
Ramsey Theory explores the underlying structure emerging in ``large enough" complex systems. For example, Frank Ramsey [1] proved that for each \( n \in \mathbb{N}  \) there is a sufficiently large \( N \in \mathbb{N}  \) such that in any red-blue coloring of the edges of the complete graph \( K_{N} \) there is a set of \( n \) vertices joined by pairwise monochromatic edges (i.e., a monochromatic clique).  This raises a natural question: given \( n, k \in \mathbb{N}  \), is there an integer \( N \in \mathbb{N} \) such that in any edge-coloring of \( K_{N}  \) in \( k \) colors there is a monochromatic clique of size \( n \)? Its positive answer is due to (authors) [2]. Analogously, in what follows, we generalize and strengthen a Ramsey-type coloring theorem of Protasov and Protasova [3] to hold for $k$ colors, where \( k \in \mathbb{N}  \).

Fix a metric space \( (X,d) \) and let \( k \in \mathbb{N}  \). A mapping \( \chi : [X]^{k} \to [k]  \) is called an \emph{isometric coloring} if \( \chi (A_1) = \chi (A_2) \) whenever \( A_1, A_2 \) is a pair of isodiametric \( k \)-element subsets of \( X \).
