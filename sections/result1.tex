\section{Main Result}


\subsection{A technical lemma} (authors) introduce the following lemma, contructing a sequence \( (x_{n}) \) with a unique property. We then use this sequence to prove the main result. Hence, for the sake of completeness, we start this section with its statement and proof.

\begin{lemma}
Let \( (X, d) \) be an infinite metric space. Then there is a sequence \( \{ x_{n}  \}_{n = 1} ^{\infty}  \) of distinct points in \( X \) such that either
\begin{enumerate}[leftmargin = 1.2cm]
	\item The sequence \( \{ d(x_1, x_{n} ) \}_{n = 1} ^{\infty}  \) is strictly monotone; or
	\item For every \( n \in \mathbb{N}  \) and \( i,j \geq n \) the distances \( d(x_{n} , x_{i} ) = d(x_{n} , x_{j} ) \).
	
\end{enumerate}
\end{lemma}
\begin{proof}
We first assume that there is a point \( x_0 \in X \) such that \( d(x_0, X) \coloneqq \{ d(x_1, x_{n} ) : x_{n} \in X \}  \) is not finite. Hence, there is a countably infinite subset \( E \subseteq X \) with \( x_0 \notin E \) and \( d(x_0, x) \neq d(x_0, y) \) for every \( x,y \in E \). We obtain from \( E \) the sequence \( \xi = \{ d(x_0, x) : x \in E \}  \). Since \( \xi \) is a sequence of reals, it has a monotone subsequence \( \{ d(x_0, x_{n} ) \}_{n=1} ^{\infty} \) whose points are distinct by construction of \( E \). Since \( d \) is a metric, \( x_{i} \neq x_{j}  \) for every \( i,j \in \mathbb{N}  \) and so \( \{ x_{n}  \}_{n = 1} ^{\infty}  \) is the desired sequence.

Otherwise, \( d(x, X) \) is finite for every \( x \in X \). Fix \( x_0 \in X \) and assume without loss of generality that \( \ell_1 \in d(x_0, X)\) is non-zero. Let \( E_1 \) be a countable subset of \( X \) with \(E_1 \subseteq  \{ x \in X : d(x_0, x) = \ell_1 \}. \) Choose \( x_1 \in E_1 \), and note that \( d(x_0, E_1) = \{ \ell_1 \} \) is a singleton set and \( x_0 \notin E_1 \). 

For \( n \geq 2 \), we choose \( x_{n}  \) and define \( E_{n}  \) inductively as follows. As above, let \( \ell_{n} \in d(x_{n-1} , X) \) be non-zero and let \( E_{n}  \) be a countable subset of \( X \) with \( E_{n} \subseteq \{ x \in X : d(x_{n-1} , x) = \ell_{n} \} . \) Again, we choose \( x_{n} \in E_{n}  \) and observe that \( d(x_{n-1} , E_{n} ) = \{ \ell_{n}  \}  \). Continuing this way, we obtain the sequence from (2).
\end{proof}

\subsection{Main result} Recall that the map \( \chi : \Gamma_{X}  \to [k] \) is called a \emph{diametric coloring} if \( \chi (A_1) = \chi (A_2) \) for every pair \( A_1, A_2 \) of compact subsets of \( X \) with \( \operatorname{diam} A_1 = \operatorname{diam} A_2  \). A subset \( A \) of \( X \) is called \emph{monochrome} if its compact subsets receive the same color; that is, there is a color \( \varphi \in [k] \) such that \( \chi (\Gamma_{A} ) = \{ \varphi  \}  \). A free ultrafilter \( \mathcal{F}  \) on \( X \) is called \emph{diametrically Ramsey} if for every diametric coloring \( \chi \) there is a monochrome set \( A  \in \mathcal{F}  \).

We are ready to prove the main result.

\begin{theorem}
Fix an infinite ultrametric space \( X \). There is a sequence \( (x_{n}) \) in \( X \) such that every free ultrafilter \( \mathcal{F}  \) containing \( (x_{n}) \) is diametrically Ramsey.
\end{theorem}
\begin{proof}
Let \( \chi \) be any diametric coloring on \( X \) and fix a free ultrafilter \( \mathcal{F}  \) containing the sequence \( (x_{n} ) \) as obtained in (lemma).

Let \( h  : (x_{n}) \to \mathbb{R^{+}} \) be a fixed map. How we define \( h \) depends on \( (x_{n}) \), so we proceed in this regard later on. Moreover, suppose \( f : \mathbb{R}^{+} \to [k] \) is any mapping satisfying \( f(h(x_{n})) = \chi (A) \) whenever \( A \in \Gamma_{X} \) is a compact subset of \( X \) with \( \operatorname{diam}A = h(x_{n}) \). Finally, we set \( c = f \circ h  \).

Write \( (x_{n}) = c ^{-1} ([k]) = \medcup_{j=1}^{k} c^{-1} (\{ j \} )  \) and observe that since \( E \in \mathcal{F}  \), (lemma) implies that there is a color \( \varphi \in [k] \) with \( c ^{-1} (\{ \varphi  \} ) \in \mathcal{F}   \). We set \( A = c ^{-1} (\{ \varphi  \} )  \) and complete the proof by showing that \( A \) is monochrome. Specifically, we show that if \( K \) is a compact subset of \( A \) then \( \chi (K) = \varphi  \). So fix such a set \( K \).

Since \( K \) is compact, there are points \( x_{i} , x_{j} \in K \) with \( i < j \) and \( d(x_{i} , x_{j}) = \operatorname{diam} K \), applying (lemma). We now consider the conditions on \( (x_{n}) \) as described in (lemma), and define \( h \) accordingly to complete the proof.

\textbf{Case 1.} We will first assume that case (1) of (lemma) holds, namely that \( (x_{n}) \) is a sequence of distinct points in \( X \) where \( \{ d(x_0, x_{n} )  \}_{n = 1} ^{\infty} \) is strictly monotone for some point \( x_0 \in X \). In this case, \( h \) will indicate the distance to \( x_0 \) from a term \( x_{n} \in E \), given by \( h (x_{n}) = d(x_0, x_{n} ) \).

If \( h  \) is strictly increasing, then we have \( d(x_0, x_{i} ) < d(x_0, x_{j} ) \). Hence (lemma) implies that \( d(x_{i} , x_{j} ) = d(x_{0} , x_{j} ) \), since \( d \) is an ultrametric. Otherwise \( d(x_0, x_{i} ) > d(x_0, x_{j} ) \) so that \( d(x_{i} , x_{j} ) = d(x_0, x_{i} ) \) using (lemma) once more. Possibly swapping the symbols \( i,j \), we assume the latter case holds. Since \( \chi \) is diametric, \( x_{j} \in A \), and \( \operatorname{diam} K = d(x_{i} , x_{j} ) = d(x_0, x_{j} )  \), we have \[c(x_{j} ) = f(d(x_0, x_{j} )) = \chi (K) = \varphi , \] as needed.

\textbf{Case 2.} We now assume that case (2) of (lemma) applies to \( (x_{n}) \). Namely, for each \( n \in \mathbb{N}  \) and \( i,j \geq n \) we have \( d(x_{n} , x_{i} ) = d(x_{n} , x_{j} ) \). Define \( h : E \to \mathbb{R}^{+}  \) by \( h(x_{n} ) = d(x_{n} , x_{n+1} )\), and note that \( h(x_{n} ) = d(x_{n} , x_{j}) \) for every \( j > n \). Again, since \( \chi \) is diametric and \( i < j\), we have \( h(x_{i} ) = d(x_{i} , x_{j} )\) and, as desired, \[ c(x_{i}) = f(d(x_{i} , x_{j} )) = \chi (K) = \varphi. \] This completes the proof, since \( A \) is \( \chi \)-monochrome in both cases.
\end{proof}
\todo{Rewrite abstract, add text throughout, more results, conclusion, checking things, clean up, acknowledgements, email, formatting, add conclusion, ultrametric inequality etc. A lot more to do!}
